\documentclass[a4paper, 11pt]{article}


%\usepackage{mathpazo}
\usepackage[onehalfspacing]{setspace}
\usepackage{graphicx}
\usepackage{amsmath, amssymb, amsfonts}
\usepackage[table]{xcolor}
\usepackage{gensymb}
\usepackage[]{booktabs}
\usepackage[utf8]{inputenc}
\usepackage{array}
\usepackage{setspace}
\usepackage{xhfill}
\usepackage{enumitem}
\usepackage{multicol}

\usepackage{geometry}
\geometry{
	total={150mm,227mm},
	left=30mm,
	top=30mm,
}

\newcommand{\cuspac}{\hspace{0.5cm}}
\newcommand{\longspac}{\hspace{1cm}}
% \newcommand{\dho}{\partial} %Use this command for partial derivative command

%%%%%%%Uncomment to remove header numbering%%%%%%%%%%
%\renewcommand{\thesection}{}
%\renewcommand{\thesubsection}{}
%\renewcommand{\thesubsubsection}{}
%%%%%%%%%%%%%%%%%%%%%%%%%%%%%%%%%%%%%%%%%%%%%%%%%%%%%


\title{Day 3 - ADCS Session 3} %Title of Document
\date{February 15th 2024} %Date of Publishing
\author{Pranav Chandra N V\\2023AAPS0013P} %Author

\begin{document}
	\maketitle
	\newpage
	\tableofcontents
	\newpage
	
	\section{Introduction}
	\begin{itemize}
		\item Estimation is, in essence, guesswork. We use multiple sensors and a bunch of nice math to figure out things like distance, velocity, rotation, magnetic field and so on.
		\item The reason we use estimation is because of the inherent uncertainties that exist in all of our measurements.
		\item In order to take measurements, we have to realize that a base reference must exist.
	\end{itemize}
	\section{States}
	\begin{itemize}
		\item In order to understand what we are in, we define the "state" of a system. A state is essentially the set of variables that describe a given system.
		\item We use time as our "base variable" and then find everything based on that.
		\item The other variabels that we consider are called dynamical variables, because they are dependent on time.
		\item This works pretty well since in a classical system, oncce you have the positions and the velocities of a system, you can figure out everything about a system.
		\item However, we still want to get all the parameters. To get this, we create a system of equations to create something called a dynamic model.
	\end{itemize}
		
	\subsection{Characterizing States}
	\begin{itemize}
			\item To characterize a state, it's often found that representing the state of the system in a vector works well.
			\subitem Vector $\neq$ direction + magnitude. We use the formal definition of a vector space, not just a vector to allow for a wider range of data storage, handling and operational capability.
			\item 
	\end{itemize}
		
	\section{Estimation}
	\begin{itemize}
		\item Given a set of data, how do we actually estimate? We generally take a bunch of estimates and then take the average. Why do we not take the median or mode though?
		\subitem The mode generally doesn't repeat due to the precision of values nowadays, the chances of two measurements being equal are minuscule. 
		\subitem We use the average since it's more efficient for data storage. We can continually compute the average using the previous average \textbf{without} storing too much data.
		\item When we have something in a constant spot, and we have multiple measurements, we get the stationary formula.
		\item When we have a moving object at a constant velocity, we can use something called the alpha-beta filter:
		\begin{equation}
			\hat{x}_{n,n} = \hat{x}_{n,n-1} + \alpha (z_n-\hat{x}_{n,n-1})
		\end{equation}
		\begin{equation}
			\begin{split}
				\hat{\dot{x}}_{n,n} = \hat{\dot{x}}_{n,n-1} + \beta\left(\frac{z_n - \hat{x}_{n,n-1}}{\Delta t}\right)
			\end{split}
		\end{equation}
		\subitem \textbf{Note: }$x_{n,n-1}$ reads as position of $n$ at time $n-1$.
		\item $\alpha$ and $\beta$ are factors that allow us to tune how much we taken in our new values over time. In cases when we don't trust our new measurements, we can lower $\alpha$, but when we trust our value is good, we can increase $\alpha$.
		\item Now you'll notice that we have $x_{n,n-1}$ which we don't have. To get this, we use the state extrapolation equation:
		\begin{equation}
			x_n = x_{n-1} + \Delta t \dot{x}_{n-1}
		\end{equation}
		\begin{equation}
			\dot{x}_{n-1} = \dot{x}_n
		\end{equation}
	\end{itemize}
	
		\subsection{Alpha-Beta-Gamma Filter}
		\begin{itemize}
			\item Previously we considered a constant velocity. Now, we can use the alpha-beta-gamma filter to linearize estimations for acceleration.
			\item They are as follows:
			\begin{equation}
				\hat{x}_{n,n} = \hat{x}_{n,n-1} + \alpha (z_n-\hat{x}_{n,n-1})
			\end{equation}
			\begin{equation}
				\begin{split}
					\hat{\dot{x}}_{n,n} = \hat{\dot{x}}_{n,n-1} + \beta\left(\frac{z_n - \hat{x}_{n,n-1}}{\Delta t}\right)
				\end{split}
			\end{equation}
			\begin{equation}
				\hat{\ddot{x}}_{n,n} = \hat{\ddot{x}}_{n, n-1} + \gamma \left(\frac{z_n - \hat{x}_{n,n-1}}{0.5 t^2}\right)
			\end{equation}
			\item Our state extrapolation factors for the same will be as follows:
			\begin{equation}
				\hat{\ddot{x}}_{n,n-1} = \hat{\ddot{x}}_{n-1,n-1}
			\end{equation}
			\begin{equation}
				\hat{\dot{x}}_{n, n-1} = \hat{\dot{a}}_{n-1, n-1} + \hat{\ddot{x}}_{n, n-1}\Delta t
			\end{equation}
			\begin{equation}
				\hat{x} = \hat{x} + \hat{\dot{x}}_{n, n-1}\Delta t + \frac{1}{2}\hat{x}_{n,n-1} \Delta t^2 
			\end{equation}
		\end{itemize}
	
	\section{State as a Random Variable}
	\begin{itemize}
		\item 
	\end{itemize}
\end{document}
