\documentclass[a4paper, 11pt]{article}


%\usepackage{mathpazo}
\usepackage[onehalfspacing]{setspace}
\usepackage{graphicx}
\usepackage{amsmath, amssymb, amsfonts}
\usepackage[table]{xcolor}
\usepackage{gensymb}
\usepackage[]{booktabs}
\usepackage[utf8]{inputenc}
\usepackage{array}
\usepackage{setspace}
\usepackage{xhfill}
\usepackage{enumitem}
\usepackage{background} %Package for adding a watermark
\usepackage[space]{grffile}
\usepackage{tikz} %An interesting package that is currently being used for numbers in circles.

\backgroundsetup{scale = 1, angle = 0, firstpage = false, opacity=0.1, contents=\includegraphics{"C:/Users/prana/Desktop/1 - College/3 Project Papers/Watermark.png"}}

\usepackage{geometry}
\geometry{
	total={150mm,227mm},
	left=30mm,
	top=30mm,
}

\newcommand{\cuspac}{\hspace{0.5cm}}
\newcommand{\longspac}{\hspace{1cm}}
% \newcommand{\dho}{\partial} %Use this command for partial derivative command

%%%%%%%Uncomment to remove header numbering%%%%%%%%%%
%\renewcommand{\thesection}{}
%\renewcommand{\thesubsection}{}
%\renewcommand{\thesubsubsection}{}
%%%%%%%%%%%%%%%%%%%%%%%%%%%%%%%%%%%%%%%%%%%%%%%%%%%%%


\title{Team Anant Probation Notes - Attitude Determination and Control Subsystem(ADCS)} %Title of Document
\date{12th February 2024} %Date of Publishing
\author{Pranav Chandra N V\\2023AAPS0013P} %Author

\begin{document}
	\maketitle
	\newpage
	
	\newpage
	\section{Control Systems}
	\begin{itemize}
		\item Deals with changing the attitude and position of the satellite.
		\item We start with our most basic definition. A \textbf{system} is any\textit{thing} that can change by virtue of the application of external and internal forces.
		\item A signal in this particular case is essentially any partcular quantity that we might be able to measure from the system.
		\item Control systems operate with differential equations.
	\end{itemize}
	\subsection{Block Representation of a System}
	\begin{itemize}
		\item We generally use block diagrams to represent systems.
		\begin{itemize}
			\item \textbf{Arrow} - Represents a signal.
			\item \textbf{Gain Block} - Gives us how the system will change the system at a certain point.
			\item \textbf{Sum Block} - Takes in two signals and outputs a single signal.
		\end{itemize}
		\item A system may have various types of gain blocks. To completely explain a system, you'd need all its arrows and gain blocks.
		\item A closed loop control system is where some kind of feedback is taken back to the input or one of the gain/sum blocks to change the output.
		\item Open loop control systems don't give us any feedback to incorporate into the output, while closed loop control systems do.
		\item They have specific representation and syntax that has to be kept in mind.
	\end{itemize}
	\subsection{The Laplace Domain}
	\begin{itemize}
		\item Analysis and solving differential equations for control systems is complicated in the standard domain.
		\item Instead, we transform points in time to points in the \textit{S} domain. 
		\item This transformed domain is known as the Laplace Domain. It simplifies calculations.
		\item By applying Laplace transformations, we obtain polynomials that are easier to solve than differential equations.
	\end{itemize}
	
	
\end{document}
